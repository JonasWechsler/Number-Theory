\documentclass{article}
\usepackage[utf8]{inputenc}
\usepackage[margin=1in]{geometry}
\usepackage[enlgish]{babel}

\usepackage{enumerate}
\usepackage{amsmath}
\usepackage{amssymb}
\usepackage{amstext}
\usepackage{amsthm}
\usepackage{wasysym}
\usepackage{array}

\newtheorem{theorem}{Theorem}[section]
\newcommand{\thf}{\rule{\textwidth}{.4pt}}
\title{Number Theory \phone}
\author{Jonas Wechsler}
\date{January 2015}

\begin{document}

	\maketitle
	%Homework: 1.1 - 1.14, pp 9-12
	\begin{theorem}
		\[ \left .
			\begin{tabular}{l}
				$a,b,c \in \mathbb{Z}$\\
				$a | b$\\
				$a | c$\\
			\end{tabular}
		\right \} \]
	$\implies a|(b+c)$\\
\end{theorem}
\begin{proof}
	$a|b \implies \exists k \in \mathbb{Z} \ni ak = b$\\
	$a|c \implies \exists j \in \mathbb{Z} \ni aj = c$\\
	$b+c=ak+aj$ by substitution\\
	$b+c=a(k+j)$ by factoring\\
	$k,j \in \mathbb{Z} \implies k+j \in \mathbb{Z}$\\
	$a|(b+c)$ by definition "$|$". $\blacksquare$ \\
\end{proof}
\begin{theorem}
	Theorem:
	\[ \left .
		\begin{tabular}{l}
			$a,b,c \in \mathbb{Z}$\\
			$a | b$\\
			$a | c$\\
		\end{tabular}
	\right \}\]
$\implies a|(b-c)$\\ 
Proof: $a|b \implies \exists k \in \mathbb{Z} \ni ak = b$\\
$a|c \implies \exists j \in \mathbb{Z} \ni aj = c$\\
$b-c=ak-aj$ by substitution\\
$b-c=a(k-j)$ by factoring\\
$k,j \in \mathbb{Z} \implies k+j \in \mathbb{Z}$\\
$a|(b+c)$ by definition "$|$". $\blacksquare$ \\
\end{theorem}\begin{theorem}
	Theorem:
	\[ \left .
		\begin{tabular}{l}
			$a,b,c in \mathbb{Z}$\\
			$a|b$\\
			$a|c$\\
		\end{tabular}
	\right \}
\]
\\
Proof: $a|b \implies \exists k \in \mathbb{Z} \ni ak = b$\\
$a|c \implies \exists j \in \mathbb{Z} \ni aj = c$\\
$akc = bc$ by algebra\\
Let $l \in \mathbb{Z} \ni l=kc$\\
$al = bc$ by substitution\\
$a|bc$ by definition\\
\end{theorem}\begin{theorem}	\emph{1.3} can be reduced to \emph{If $a|b$, then $a|bc$.} \emph{1.3} can also be reworked to \emph{If $a|b$ and $a|c$, then $a^2|bc$.}
	\begin{enumerate}
		\item
			$a|b$\\
			$an=b$ by definition\\
			$anc=bc$ by algebra\\
			$\exists m \in \mathbb{Z} \ni am=bc$ and $m=nc$\\
			$a|bc$ by definition\\
		\item
			$a|b$ and $a|c$\\
			$an=b$ by def\\
			$am=c$ by def\\
			$anam=bc$ by algebra\\
			$a^2nm=bc$ by algebra\\
			$a^2|bc$ by definition
	\end{enumerate}
\end{theorem}\begin{theorem} 
	Let $a,b,c \in \mathbb{Z}.$ If $a|b$, then $a|b^n$.\\
	$a|b$\\
	$an=b$ by def\\
	$anb=bb$ by algebra\\
	$\exists m \in \mathbb{Z} \ni am=bb$ and $m=nb$\\
	$am=b^2$ by algebra\\
\end{theorem}\begin{theorem}
	$a|b$\\
	$an=b$ by definition\\
	$anc=bc$ by algebra\\
	$\exists m \in \mathbb{Z} \ni am=bc$ and $m=nc$\\
	$a|bc$ by definition\\
\end{theorem}\begin{theorem}
	\begin{enumerate}
		\item[1]
			$45 \equiv 9~\pmod{4}$?\\
			$\exists m \in \mathbb{Z}|m(4) + 9 = 45$\\
			$4m + 9 = 45$\\
			$4m = 36$\\
			$m = 9$\\
			Yes
		\item[2]
			$37 \equiv 2~\pmod{5}$?\\
			$\exists m \in \mathbb{Z}|m(5) + 2 = 37$\\
			$5m + 2 = 37$\\
			$5m = 35$\\
			$m = 7$\\
			Yes
		\item[3]
			$37 \equiv 3~\pmod{5}$?\\
			$\exists m \in \mathbb{Z}|m(5) + 3 = 37$\\
			$5m + 3 = 37$\\
			$5m = 34$\\
			No
		\item[4]
			$37 \equiv -3~\pmod{5}$?\\
			$\exists m \in \mathbb{Z}|m(5) - 3 = 37$\\
			$5m - 3 = 37$\\
			$5m = 40$\\
			$m = 8$\\
			Yes
	\end{enumerate}
\end{theorem}\begin{theorem}
	\begin{enumerate}
		\item[1]
			$m \equiv 0~\pmod{3}$\\
			$\exists n \in \mathbb{Z}|n(3) + 0 = m$\\
			$3n + 0 = m$\\
		\item[2]
			$m \equiv 1~\pmod{3}$\\
			$\exists n \in \mathbb{Z}|n(3) + 1 = m$\\
			$3n + 1 = m$\\
		\item[3]
			$m \equiv 2~\pmod{3}$\\
			$\exists n \in \mathbb{Z}|n(3) + 2 = m$\\
			$3n + 2 = m$\\
		\item[4]
			$m \equiv 3~\pmod{3}$\\
			$\exists n \in \mathbb{Z}|n(3) + 3 = m$\\
			$3n + 3 = m$\\
		\item[5]
			$m \equiv 4~\pmod{3}$\\
			$\exists n \in \mathbb{Z}|n(3) + 4 = m$\\
			$3n + 4 = m$\\
	\end{enumerate}
\end{theorem}\begin{theorem}
	\begin{enumerate}
		\item[1]
			$n \equiv b \pmod{k}$\\
			$k|(n-b)$ def of mod\\
			$\exists ak = n-b$ def of "$|$"\\
			$ak+b=n$\\
	\end{enumerate}
\end{theorem}\begin{theorem}
	%$ a \equiv b \mod{c} \implies c|(a-b) \implies ck = a-b$
	%$ a \equiv a \mod{n} \iff n|(a-a) \iff nk = a-a$
	Theorem:
	\[\left . 
		\begin{tabular}{l}
			$a,n \in \mathbb{Z}$\\
			$n~>~0$\\
		\end{tabular}
	\right \}\]
$\implies a \equiv a \mod{n}$\\
Proof:	$ n0=0$ by algebra\\
$ a-a = 0$ by algebra\\
$\exists k \in \mathbb{Z} \ni k=0$\\
$ nk = 0$ by substitution\\
$ nk = a-a$ by substitution\\
$ n|a-a$ by definition\\
$ a \equiv a \mod{n}$ by definition $\blacksquare$\\
\end{theorem}\begin{theorem}
	Theorem:
	\[\left .
		\begin{tabular}{l}
			$a,b,n \in \mathbb{Z}$\\
			$n > 0$\\
			$a \equiv b \mod{n}$
		\end{tabular}
	\right \}\]
$\implies b \equiv a \mod{n}$\\
Proof: $a \equiv b \mod{n}$\\
$n|(a-b)$ by definition\\
$\exists k \in \mathbb{Z} \ni nk=a-b$\\
$-nk=-a+b$ by algebra\\
$-kn=b-a$ by algebra\\
$n|(b-a)$ by definition\\
$b \equiv a \mod{n}$ by definition $\blacksquare$\\
\end{theorem}\begin{theorem}
	Theorem:
	\[\left .
		\begin{tabular}{l}
			$a,b,c,n \in \mathbb{Z}$\\
			$n > 0$\\
			$a \equiv b \mod{n}$\\
			$b \equiv c \mod{n}$\\
		\end{tabular}
	\right \}\]
$\implies a \equiv c \mod{n}$\\
Proof: $a \equiv b \mod{n}$ by definition\\
$b \equiv c \mod{n}$ by definition\\
$n|(a-b)$ by definition\\
$n|(b-c)$ by definition\\
$nk=b-c \land nj=a-b$\\
$nk-nj=(b-c)+(a-b)$ by algebra\\
$n(k-j)=b-c+a-b$ by algebra\\
$n(k-j)=a-c$ by algebra
$n|(a-c)$ by definition\\
$a \equiv c \mod{n}$ by definition $\blacksquare$\\
\end{theorem}\begin{theorem}	
	Theorem:
	\[
		\left . .
		\begin{tabular}{l}
			$a,b,c,d,n \in \mathbb{Z}$\\
			$n > 0$\\
			$a \equiv b \mod{n}$\\
			$c \equiv d \mod{n}$\\
		\end{tabular}
	\right \}
\]
$\implies a+c \equiv b+d \mod{n}$\\
Proof:$a \equiv b \mod{n}$\\
$c \equiv d \mod{n}$\\
$n|(a-b)$ by definition\\
$n|(c-d)$ by definition\\
$n|(a-b)+(c-d)$ by Thm 1.1\\
$n|(a+c-b-d)$ by algebra\\
$n|((a+c)-(b+d))$ by Algebra\\
$a+c \equiv b+d \mod{n}$ by definition $\blacksquare$\\
\end{theorem}\begin{theorem}
	Theorem:
	\[\left .
		\begin{tabular}{l}
			$a,b,c,d,n \in \mathbb{Z}$\\
			$n > 0$\\
			$a \equiv b \mod{n}$\\
			$c \equiv d \mod{n}$\\
		\end{tabular}
	\right \}\]
$\implies a-c \equiv b-d \mod{n}$\\
Proof:$a \equiv b \mod{n}$\\
$c \equiv d \mod{n}$\\
$n|(a-b)$ by definition\\
$n|(c-d)$ by definition\\
$n|((a-b)-(c-d))$ by Thm 1.2\\
$n|((a-c)-(b-d))$ by Algebra\\
$a-c \equiv b-d \mod{n}$ by definition $\blacksquare$\\
\end{theorem}\begin{theorem}
	Theorem:
	\[\left .
		\begin{tabular}{l}
			$a,b,c,d,n \in \mathbb{Z}$\\
			$n > 0$\\
			$a \equiv b \mod{n}$\\
			$c \equiv d \mod{n}$\\
		\end{tabular}
	\right \}\]
$\implies ac \equiv bd \mod{n}$\\
Proof:$a \equiv b \mod{n}$\\
$c \equiv d \mod{n}$\\
$n|(a-b)$ by definition\\
$n|(c-d)$ by definition\\
$n|a$ and $n|b$ and $n|c$ and $n|d$\\
$n|ac$ by Thm 1.3\\
$n|ad$ by Thm 1.3\\
$n|((ac)-(bd)$ by Thm 1.2\\
$ac \equiv bd \mod{n}$ by definition $\blacksquare$\\
\end{theorem}\begin{theorem}
	Theorem:
	\[\left .
		\begin{tabular}{l}
			$a,b,n \in \mathbb{Z}$\\
			$n > 0$\\
			$a \equiv b \mod{n}$\\
		\end{tabular}
	\right \}\]
$\implies a^2 \equiv b^2 \mod{n}$\\
Proof: $a \equiv b \mod{n}$\\
$n|(a-b)$ by defintion\\
$n|a$ and $n|b$ by definition\\
$n|a^2$ and $n|b^2$ by Thm 1.6\\
$n|(a^2 - b^2)$ by Thm 1.2\\
$a^2 \equiv b^2 \mod{n}$ by defnition $\blacksquare$\\
\end{theorem}\begin{theorem}
	Theorem:
	\[\left .
		\begin{tabular}{l}
			$a,b,n \in \mathbb{Z}$\\
			$n > 0$\\
			$a \equiv b \mod{n}$\\
		\end{tabular}
	\right \}\]
$\implies a^3 \equiv b^3 \mod{n}$\\
Proof: $a \equiv b \mod{n}$\\
$n|(a-b)$ by defintion\\
$n|a$ and $n|b$ by definition\\
$n|a^2$ and $n|b^2$ by Thm 1.6\\
$n|a^3$ and $n|b^3$ by Thm 1.6\\
$n|(a^3 - b^3)$ by Thm 1.2\\
$a^3 \equiv b^3 \mod{n}$ by defnition $\blacksquare$\\
\end{theorem}\begin{theorem}
	Theorem:
	\[\left .
		\begin{tabular}{l}
			$a,b,k,n \in \mathbb{Z}$\\
			$n > 0$\\
			$k > 1$\\
			$a \equiv b \mod{n}$\\
			$a^{k-1} \equiv b^{k-1} \mod{n}$\\
		\end{tabular}
	\right \}\]
$\implies a^k \equiv b^k \mod{n}$\\
Proof: $a \equiv b \mod{n}$\\
$a^{k-1} \equiv b^{k-1} \mod{n}$\\
$n|a$ and $n|b$ and $n|a^{k-1}$ and $n|b^{k-1}$ by definition\\
$n|(a^{k-1}a)$ and $n|(a^{k-1}a)$ by Thm 1.3\\
$n|a^k$ and $n|b^k$ by algebra\\
$n|(a^k-b^k)$ by Thm 1.2\\
$a^k \equiv b^k \mod{n}$ by defintion\\
\end{theorem}\begin{theorem}
	Theorem:
	\[\left .
		\begin{tabular}{l}
			$a,b,k,n \in \mathbb{Z}$\\
			$n > 0$\\
			$k > 0$\\
			$a \equiv b \mod{n}$\\
		\end{tabular}
	\right \}\]
$\implies a^k \equiv b^k \mod{n}$\\
Proof: $a \equiv b \mod{n}$\\
$n|a$ and $n|b$\\
$n|(a^k)$ and $n|(a^k)$ by Thm 1.6\\
$n|(a^k-b^k)$ by Thm 1.2\\
$a^k \equiv b^k \mod{n}$ by defintion\\
\end{theorem}\begin{theorem}
	$ a \equiv b \mod{c} \implies c|(a-b) \implies ck = a-b$
	\begin{enumerate}
		\item
			$12 \equiv 2 \mod{5} \implies 5k = 12-2$\\
			$20 \equiv 5 \mod{5} \implies 5k = 20-5$\\
			$32 \equiv 7 \mod{5} \implies 5k = 32-7$
		\item
			$12 \equiv 2 \mod{5} \implies 5k = 12-2$\\
			$20 \equiv 5 \mod{5} \implies 5k = 20-5$\\
			$-8 \equiv -3 \mod{5} \implies 5k = -8+3$
		\item
			$12 \equiv 2 \mod{5} \implies 5k = 12-2$\\
			$20 \equiv 5 \mod{5} \implies 5k = 20-5$\\
			$240 \equiv 10 \mod{5} \implies 5k = 240-10$
		\item
			$12 \equiv 2 \mod{5} \implies 5k = 12-2$\\
			$144 \equiv 4 \mod{5} \implies 5k = 144-4$
		\item
			$12 \equiv 2 \mod{5} \implies 5k = 12-2$\\
			$1728 \equiv 8 \mod{5} \implies 5k = 1728-8$
		\item
			% nk = a-b
			% nk = a^k - b^k
			$9 \equiv 5 \mod{2} \implies 2k = 9-5$\\
			$9^{4-1} \equiv 5^{4-1} \mod{2} \implies 2k = 729-125$\\
			$9^4 \equiv 5^4 \mod{2} \implies 2k = 6561-625$
		\item
			$12 \equiv 2 \mod{5} \implies 5k = 12-2$\\
			$12^k \equiv 2^k \mod{5} \implies 5k = 12^k-2^k$
	\end{enumerate}
\end{theorem}\begin{theorem}
	Theorem:
	\[\left .
		\begin{tabular}{l}
			$a,b,c,n \in \mathbb{Z}$\\
			$ac \equiv bc \mod{n}$\\
		\end{tabular}
	\right \}\]
$\implies a \equiv b \mod{n}$\\
Proof: $ac \equiv bc \mod{n}$\\
$n|(ac-bc)$ by definition\\
$nk = ac - bc$ by definition\\
$n \frac{k}{c} = a-b$ by algebra\\
If $c|k$, then we can conclude that $\exists j \in \mathbb{Z} \ni j=\frac{k}{c}$\\
In this case, $nj = a-b$ by substitution, $n|(a-b)$ by definition, and $a \equiv b \mod{n}$ by definition.\\
\end{theorem}\begin{theorem}
	Theorem:
	\[\left .
		\begin{tabular}{l}
			$n,m,a_i \in \mathbb{N}$\\
			$0 \leq a_i \leq 9$\\
			$n = a_ka_{k-1}...a_1a_0$\\
			$m = a_k+a_{k-1}...+a_1+a_0$\\
		\end{tabular}
	\right \}\]
	\end{theorem}
	$\implies n \equiv m \mod{3}$\\
	\begin{proof}
		$n-m=(a_0$
		$3|(n-m)$\\
		$3k=n-m$\\
	\end{proof}
	\begin{theorem}

	\end{theorem}
	\begin{theorem}%1.25
		
	\end{theorem}
	%Homework: through #35
\end{document}

