\documentclass{article}
\usepackage[utf8]{inputenc}
\usepackage[margin=1in]{geometry}
\usepackage{enumerate}
\usepackage{amsmath}
\usepackage{amssymb}
\usepackage{amstext}
\usepackage{array}
\newcommand{\thf}{\rule{\textwidth}{.4pt}}

\title{Day one lecture}
\author{Jonas Wechsler}
\date{January 5, 2015}

\begin{document}

	\maketitle
	\section{Proofs!}
	\begin{tabular}{l l}
		Definition: & $\mathbb{N}$ : {1,2,3...}\\
		& $\mathbb{Z}$ : {... -3,-2,-1,0,1,2,3...}\\
		\\
		Definition: & If $a, d \in \mathbb{Z}$, then\\
		& $d|a$ ("d divides a") if $\exists k \in \mathbb{Z}$ $\ni a = kd$\\
		\\
		Definition: & If $a, b, n \in \mathbb{Z}$ and $n > 0$, then \\
		& $a \equiv b$ mod $n$ if $n|(a-b)$\\ 
		& "a is congruent to b modulo n"
	\end{tabular}
	\\\vspace{1 mm}\\
	\begin{tabular}{l l}
		Theorem: & \[
			\left 
			\begin{tabular}{l}
				$n \in \mathbb{Z}$\\
				$6 | n$\\
			\end{tabular}
		\right \}
	\] $\implies 3|n$\\ 
\end{tabular}\\
\vspace{1 mm}\\
$a\equiv b mod(n) \iff \exists m|mn+b=a$\\
\vspace{1 mm}\\
%Homework: 1.1 - 1.14, pp 9-12
%\section{Day 1 homework}
%\begin{enumerate}
%	\item[1.1]	Let $a,b,c \in \mathbb{Z}.$ If $a|b$ and $a|c$, then $a|(b+c)$
%	\item[1.2]	Let $a,b,c \in \mathbb{Z}.$ If $a|b$ and $a|c$, then $a|(b-c)$
%	\item[1.3]	Let $a,b,c \in \mathbb{Z}.$ If $a|b$ and $a|c$, then $a|bc$
%\end{enumerate}\\
%\section{Day ??? Homework}
\begin{tabular}{l l}
	\emph{Well ordering axiom} & In any subset of \mathbb{N} there exists a smallest element.\\
\end{tabular}
\begin{theorem}: & $\forall n \in \mathbb{N} \exists k \in \mathbb{N} \ni |7k-n|<7$
\end{theorem}\begin{proof}
Define $s={7i|i \in \mathbb{N} 7i>n}$\\
By Well Ordering Axiom S contains a smallest element, call it $7k$(for some $k \in \mathbb{N}$).\\
$7k-n$ must be less than 7, else\\
$7(k-1)=7k-7>n$ and $7k$ is not the smallest element of S. $\blacksquare$

\begin{definition}
	$a,b,d \in \mathbb{Z}$\\
	$d|a \land d|b$\\
	$d$ is a common divisor of $a$ and $b$\\
	If $x>d \leftrightarrow x \nmid a \lor x \nmid b$, then
	d is the \emph{greatest common divisor}\\
	denoted gcd(a,b)=d or simply (a,b)=d\\
	If (a,b) = 1, then a and b are said to be \emph{relatively prime} or \emph{coprime}.\\
	Ex (8,15)=1 8 and 15 are coprime\\

\end{definition}
\begin{tabular}{l l}
	\emph{The Division Algoirthm} & Given any pair of natural numbers $n \land m \exists$ a unique pair of integers q (for quotient) and r (for remainder) satisying $m=nq-r$ with $0 \leq r \leq n-1$\\
\end{tabular}
\end{document}

